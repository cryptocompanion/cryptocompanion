\newpage
\subsection{Pseudorandom generators (PRG)}
A pseudorandom generator is an algorithm that takes random input and produces longer pseudorandom output.

\begin{syntax}[Pseudorandom generator (PRG)]
  A pseudorandom generator with stretch $s$ is a deterministic function
  \[g:\bin^*\rightarrow\bin^*\]
  that can be computed in deterministic polynomial time, which satisfies the correctness criteria
  \begin{align*}
    \forall \lambda\in\NN                            & :\ s(\lambda)\geq 1              \\
    \forall \lambda\in\NN\; \forall x\in\bin^\lambda & :\ \abs{g(x)}=\lambda+s(\lambda) \\
  \end{align*}
\end{syntax}

Pseudorandom generators are not as natural to define in terms of security games as other primitives. We therefore define it first by a \emph{security experiment}, before giving a security game formulation for completeness.

\begin{security}[Pseudorandomness]
  \vspace{5mm}
  Let ${\mathsf{Exp}}_{g,s,\adv}^{\mathsf{PRG},0}(1^\lambda)$ and ${\mathsf{Exp}}_{g,\adv}^{\mathsf{PRG},1}(1^\lambda)$ be the security experiments defined by
  \begin{center}
    \begin{pchstack}
      \procedure{${\mathsf{Exp}}_{g,s,\adv}^{\mathsf{PRG},0}(1^\lambda)$}{
        x\sample\bin^\lambda\\
        y\gets g(x)\\
        d^*\sample\adv(1^\lambda,y)\\
        \pcreturn d^*}
      \pchspace
      \procedure{${\mathsf{Exp}}_{s,\adv}^{\mathsf{PRG},1}(1^\lambda)$}{
        \\
        y\sample\bin^{\lambda+s(\lambda)}\\
        d^*\sample\adv(1^\lambda,y)\\
        \pcreturn d^*}
    \end{pchstack}
  \end{center}
  A PRG $g$ is pseudorandom if for all PPT adversaries $\adv$ the difference
  \[ \big|\prob{{\mathsf{Exp}}_{g,s,\adv}^{\mathsf{PRG},0}(1^\lambda) = 1}
    - \prob{{\mathsf{Exp}}_{s,\adv}^{\mathsf{PRG},1}(1^\lambda) = 1}\big| \]
  is negligible in $\lambda$.
\end{security}


\begin{security}[PRG]
  Let $\textsf{Gprg}_g^0$ and $\textsf{Gprg}_g^1$ be the games
  \begin{codebox}
    \begin{center}
      \begin{pchstack}
        \pchspace
        \begin{pcvstack}
          \underline{\underline{$\textsf{Gprg}_g^0$}}\\
          \\
          \procedure{Package Parameters}{
            \lambda\text{:} \< \quad \text{security parameter}\\
            s(\lambda)\text{:} \< \quad \text{length-expansion}\\
            g\text{:} \< \quad \text{function},\,\abs{g(x)}=\abs{x}+s(\abs{x})
          }
          \pcvspace
          \procedure{Package State}{
            y\text{:} \< \quad \text{image value}
          }
          \pcvspace
          \procedure{$\O{SAMPLE}()$}{
            \pcassert y = \bot\\
            x\sample\bin^\lambda\\
            y \gets g(x)\\
            \pcreturn y}
        \end{pcvstack}
        \pchspace
        \begin{pcvstack}
          \underline{\underline{$\textsf{Gprg}^1$}}\\
          \\
          \procedure{Package Parameters}{
            \lambda\text{:} \< \quad \text{security parameter}\\
            s(\lambda)\text{:} \< \quad \text{length-expansion}\\
          }
          \pcvspace
          \procedure{Package State}{
            y\text{:} \< \quad \text{random value}
          }
          \pcvspace
          \procedure{$\O{SAMPLE}()$}{
            \pcassert y = \bot\\
            \\
            y \sample \bin^{\lambda+s(\lambda)}\\
            \pcreturn y}
        \end{pcvstack}
      \end{pchstack}
    \end{center}
  \end{codebox}
  A pseudorandom generator $g$ is PR-secure if the real and ideal games $\textsf{Gprg}_g^0$ and $\textsf{Gprg}^1$ are computationally indistinguishable, i.e., that is, the advantage
  \[\mathbf{Adv}_{\adv}^{
    \M{Gprg}_g^0,
    \M{Gprg}^1}
    (\lambda):=\abs{\prob{1=\adv\circ \M{Gprg}_g^0}-\prob{1=\adv\circ \M{Gprg}^1}}\]
  is negligible in $\lambda$.
\end{security}
\begin{remark}
  In the course, when saying that \emph{$g$ is a PRG}, we mean a \emph{$g$ is a PR-secure PRG}.
\end{remark}