A cryptographic primitive attempts to capture the core idea of a cryptographic tool, often one that is used in many different protocols. In applied cryptography, primitives are concrete functions that are trusted to perform an action and to be secure under some definition. Our approach is more abstract, so we give formal definitions of primitives. The hope is that they capture the important qualities of real-life primitives, and that those real-life primitives in turn fulfill the mathematical definitions.

This section uses the definitions from Section \ref{section:definitions} to define primitives, which are foundational to cryptography as explained in Section \ref{section:foundations}. We define a primitive in two parts: \emph{syntax} and \emph{security}. The first describes the basic functionality of the primitive, and the second the way in which it is secure. This separation is useful, since it allows us to have multiple different definitions of security for the same syntax.

Each subsection starts with a short motivation for the primitive in question. Afterwards, the syntax is defined using mathematical notation. The syntax contains descriptions of the inputs and outputs of the functions involved, and possibly a correctness criterion. When defining the syntax of a primitive, we often won't give a full function definition. We will instead leave the exact binary representations implicit, and represent the input and output by symbols relevant to the primitive in question. When we do this, the output can always also be $\bot$.

After defining the syntax, we define what security means for the primitive. This most often means defining a security game, as explained in Section \ref{ssec:security-games}. However, the first two primitives (OWFs and PRGs) are not as natural to think of as security games, so we first give an alternative security definition.

This section is best used as a reference for the definitions of different primitives. Each subsection is independent of the others, so you can go straight to the section giving the definition you are interested in. This section uses the notations of Section \ref{section:definitions}, so you should return there if something is unclear. The definitions here are quite sparse: further elaboration on modelling choices can be found in Section \ref{section:foundations}, especially Section~\ref{ssec:security-foundations}.